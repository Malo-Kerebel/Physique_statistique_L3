\documentclass[12pt,a4paper]{report}
\usepackage[utf8]{inputenc}
\usepackage[french]{babel}
\usepackage[T1]{fontenc}
\usepackage{amsmath}
\usepackage{amsfonts}
\usepackage{amssymb}
\usepackage{graphicx}
\author{Malo Kerebel}

\usepackage{systeme}

\usepackage[mathscr]{euscript}

\usepackage{hyperref}
\hypersetup{
    colorlinks,
    citecolor=black,
    filecolor=black,
    linkcolor=black, urlcolor=black
}

\newcommand{\Ens}[1]{\mathbb{#1}}
\newcommand{\ens}[1]{\mathbb{#1}}
\newcommand{\fphi}{\quad \forall \varphi \in \mathcal{D}}
\newcommand{\D}{\ensuremath{\mathcal{D}}}

\input cyracc.def
\font\tencyr=wncysc10
\def\cyr{\tencyr\cyracc}
\def\dc{\mbox{\cyr SH}}


\begin{document}

\subsection{Exercice 5}

\[
	F(v_x, v_y, v_z) = f_1(v_x) \cdot f_2(v_y) \cdot f_3(v_z)
\]

Si la distribution des vitesse est isotrope on a \(f_1 = f_2 = f_3 = f\), donc F peut se réécrire
\[
	F(v_x, v_y, v_z) = f(v_x) \cdot f(v_y) \cdot f(v_z)
\]

Dans un gaz parfait, l'énergie cinétique permet de classer statistiquement les molécules :
\[
	F(v^2_x + v^2_y + v^2_z) = f(v_x) \cdot f(v_y) \cdot f(v_z)
\]

La fonction de distribution des vitesses suivant x est :
\[
	f(v_x) = A_x \cdot e^{-\alpha v_x^2}
\]
Sachant qu'il y a isotropie : \(A_x = A_y = A_z = A\)

En point de départ on prend \(A = \sqrt{\dfrac{\alpha}{\pi}}\)

\begin{align*}
	\left( \int_{-\infty}^{+\infty} \sqrt{\dfrac{\alpha}{\pi}} e^{-\alpha v_x^2} dv_x\right)^2 &= 1\\
	\dfrac{\alpha}{\pi} \cdot \left( \int_{-\infty}^{+\infty} e^{-\alpha v_x^2} dv_x\right)^2 &= 1\\
\end{align*}

On fait un changement de variable de coordonnées cartésiennes à polaire :
\[
	\left( \int_{-\infty}^{+\infty} A e^{-\alpha v_x^2}\right) \left( \int_{-\infty}^{+\infty} A e^{-\alpha v_y^2} \right) = 1
\]
\begin{align*}
	A^2 \cdot \left( \int_0^{2\pi} \int_0^{+\infty} v \cdot e^{-\alpha v^2} dv d\theta \right) &= 1\\
	2\pi \cdot A^2 \cdot \left( \int_0^{+\infty} v \cdot e^{-\alpha v^2} dv \right) &= 1\\
	2\pi \cdot A^2 \cdot \left[ \dfrac{-e^{-\alpha v^2}}{2\alpha} \right]_0^{+\infty} &= 1\\
	2\pi \cdot A^2 \cdot \left( \dfrac{1}{2\alpha}\right) &= 1 \Leftrightarrow A =  \sqrt{\dfrac{\alpha}{\pi}}
\end{align*}

\[
	\langle v_x^2 \rangle = \int_{-\infty}^{+\infty} A_x v_x^2 e^{-\alpha v_x^2} dv_x = -A_x \dfrac{d}{d\alpha}\left( \int_{-\infty}^{+\infty} e^{-\alpha v_x^2} dv_x \right)
\]
où \(\left( \int_{-\infty}^{+\infty} e^{-\alpha v_x^2} dv_x \right) = \frac{1}{A} = \sqrt{\frac{\pi}{\alpha}}\)
Donc :
\[
	-A_x \dfrac{d}{d\alpha}\left( \int_{-\infty}^{+\infty} e^{-\alpha v_x^2} dv_x \right) = -A_x \dfrac{d}{d\alpha} \left(\sqrt{\frac{\pi}{\alpha}}\right) = -A_x \dfrac{d}{d\alpha} \left(\sqrt{\pi} \cdot \alpha^{-\frac{1}{2}}\right)
\]
\begin{align*}
	-A_x \dfrac{d}{d\alpha}\left( \int_{-\infty}^{+\infty} e^{-\alpha v_x^2} dv_x \right) &= -\dfrac{1}{2} \cdot A_x \cdot \sqrt{\pi} \cdot \alpha^{-\frac{3}{2}}\\
	\langle v^2 \rangle = \dfrac{k_B T}{m} &= -\dfrac{1}{2} \cdot A \cdot \sqrt{\pi} \cdot \alpha^{-\frac{3}{2}}\\
	&= -\dfrac{1}{2} \cdot \sqrt{\alpha} \sqrt{\dfrac{1}{\pi}} \cdot \sqrt{\pi} \cdot \alpha^{-\frac{3}{2}}\\
	&= -\dfrac{1}{2\alpha}\\
	\Leftrightarrow A &= \left( \dfrac{m}{2\pi k_B T} \right)^{3/2}
\end{align*}

Finalement on a la densité de probabilité, la distribution Maxwellienne des vitesses:
\[
	G(v) = 4\pi v^2 \cdot \left( \dfrac{m}{2\pi k_B T}\right)^{3/2} \cdot e^{-\dfrac{mv^2}{2k_B T}}
\]

\begin{align*}
	\langle v \rangle &= \int_0^{+\infty} v \cdot G(v) dv\\
	&=  \int_0^{+\infty} C \cdot	v^3 \cdot e^{-\dfrac{mv^2}{2k_B T}} dv\\
	&= \sqrt{\dfrac{8k_B T}{\pi m}}
\end{align*}
L'intégration par partie fonctionne pour faire ce calcul.

Mais on va le faire avec la fonction \(\Gamma\) d'Euler. Elle s'écrit :
\[
	\Gamma(z) = \int_0^{+\infty} t^{z-1} \cdot e^{-t} dt
\]
\[
	\Gamma(z + 1) = z \cdot \Gamma(z) 
\]

On met \(\langle v \rangle\) sous la forme d'une fonction d'Euler :

\begin{align*}
	\langle v \rangle &= \int_0^{\infty} C \cdot v^3 \cdot e^{-\dfrac{mv^2}{2k_B T}} dv\\
	&= \dfrac{C}{2a} \int_0^{\infty} te^{-t} dt = \dfrac{C}{2a^2} \Gamma(2)\\
	a &= \dfrac{m}{2k_B T}\\
	C &= 4\pi \cdot \left( \dfrac{m}{2k_B T}\right)^\frac{3}{2}
\end{align*}

Or \(\Gamma(2) = \Gamma (1 + 1) = 1 \cdot \Gamma(1) = \int_0^{+\infty} e^{-t} dt = 1\) et une fois qu'on remplace dans l'expression on retrouve le même résultat que précedemment.

Pour trouver la vitesse quadratique moyenne on utilise de nouveau la fonction d'Euler.
\begin{align*}
	\langle v^2 \rangle &= \int_0^{+\infty} C \cdot v^4 \cdot e^{-\dfrac{mv^2}{2k_B T}} dv\\
	&= C \int_0^{+\infty} \dfrac{t^2}{a^2}e^{-t} \dfrac{dt}{2av}\\
	&= C \int_0^{+\infty} \dfrac{t^2}{a^3} \sqrt{\dfrac{a}{t}} e^{-t} dt\\
	&= \dfrac{C}{2a^{5/2}} \int_0^{+\infty} t^{3/2} e^{-t} dt\\
	&= \dfrac{C}{2a^{5/2}} \Gamma\left(\dfrac{5}{2}\right)\\
	\Gamma\left(\dfrac{5}{2}\right) &= \dfrac{3\sqrt{\pi}}{4}\\
\end{align*}
\begin{align*}
	\langle v^2 \rangle &= \dfrac{3\sqrt{\pi}}{4} \cdot \dfrac{C}{2a^{5/2}}\\
	&= \dfrac{3}{4} \dfrac{\sqrt{\pi} \cdot \pi \cdot \left( \dfrac{m}{2k_B T}\right)^\frac{3}{2}}{\dfrac{m}{2k_B T}^{5/2}}\\
	&= \dfrac{3}{4} \dfrac{\sqrt{\pi} \cdot \pi}{m\dfrac{1}{2k_B T}}\\
	&= \dfrac{3}{2} \dfrac{\sqrt{\pi} \cdot \pi k_B T}{m}\\
\end{align*}



\end{document}