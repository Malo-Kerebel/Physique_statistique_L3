\documentclass[12pt,a4paper]{article}
\usepackage[utf8]{inputenc}
\usepackage[french]{babel}
\usepackage[T1]{fontenc}
\usepackage{amsmath}
\usepackage{amsfonts}
\usepackage{amssymb}
\usepackage{graphicx}
\author{Malo Kerebel}

\usepackage{systeme}

\usepackage[mathscr]{euscript}

\usepackage{hyperref}
\hypersetup{
    colorlinks,
    citecolor=black,
    filecolor=black,
    linkcolor=black, urlcolor=black
}

\newcommand{\Ens}[1]{\mathbb{#1}}
\newcommand{\ens}[1]{\mathbb{#1}}
\newcommand{\fphi}{\quad \forall \varphi \in \mathcal{D}}
\newcommand{\D}{\ensuremath{\mathcal{D}}}

\input cyracc.def
\font\tencyr=wncysc10
\def\cyr{\tencyr\cyracc}
\def\dc{\mbox{\cyr SH}}


\begin{document}

\section{Dénombrement d'états}

\subsection{particules discernables}
Les seuls type de combinaisons possibles sont :
\begin{align*}
	1-2-3\\
	1-1-4\\
	2-2-2
\end{align*}

Pour \(1-2-3\) on peut mettre dans l'ordre que l'on veut, il y a donc 3! = 6 combinaisons possibles. Pour \(1-1-4\), on peut mettre le 4 à chaque particule, il y a donc 3 combinaisons possibles. Pour \(2-2-2\), on ne peut mélanger les énergies, il n'y a donc qu'une seule combinaison possible. Au total il y a donc 10 combinaisons possibles.

\subsection{Boson}

Pour des bosons de spin nul, ils ne sont plus discernables. Donc il n'y a plus que la possibilité 1-2-3, 1-1-4 et 2-2-2, avec une seule possibilité pour chaque donc il n'y a que 3 possibilités.

\subsection{fermions}

On utilise la formule :
\[
	\Omega = \sum_{conf} \prod_r \dfrac{g_r!}{n_r!(g_r-n_r)!}
\]
exemple :

Pour \(1-1-4\), pour 1, on a \(n_r = 2\) car il y a deux fermions à ce niveau d'énergie.
\[
	\Omega^{(1-1-4)} = \left( \dfrac{2!}{2! (2-2)!} \right)^2 \cdot \dfrac{2!}{1!(2-1)!} = 1^2 \cdot 2 = 2
\]
Pour (1-2-3) on a \(n_r = 1\) car il n'y a que 1 fermions par niveau d'énergie. 
\[
	\Omega^{(1-2-3)} = \left( \dfrac{2!}{2!(2-2)!} \right)^3 = 8
\]

Il y a donc 10 combinaisons possibles

\section{Particules dans un puits infinis}

\subsection{énergies des états stationnaires}

\[
	E = (n_1^2 + n_2^2 +n_3^2)\varepsilon_0
\]
Avec \(n_1, n_2, n_3 > 0\)

\subsection{\(E = 33\varepsilon_0\)}

Il y a 6 micro-états accessibles. 4-4-1 et 5-2-2

\subsection{Particules indiscernables}

\subsubsection{s = 0}
La formule est :

\[
	\Omega^B = \prod_r \dfrac{(n_r + g_r -1)!)}{n_r!-g_r-1)!}
\]
Avec \(g_r = 1\) car les états sont indiscernables.
Ici on a :
\[
	\Omega^{(1-4-4)} = \Omega^{(5-2-2)} = \dfrac{(1+1-1)!}{1!0!} \times \dfrac{(2+1-1)!}{2!0!} = 1
\]

Il y a donc deux combinaisons

\subsubsection{s = 1}

ici \(g_r = 3\)

\begin{align*}
	\Omega^{(1-4-4)} = \Omega^{(5-2-2)} &= \dfrac{(1+3-1)!}{1!(3-1)!} \times \dfrac{(2+3-1)!}{2!(3-1)!}\\
	&= \dfrac{3!}{2!} \times \dfrac{4!}{2!2!}\\
	&= 3 \times \dfrac{24}{4}\\
	&= 3 \times 6 = 18
\end{align*}

Il y a donc 36 configurations.

\subsubsection{s = 1/2}

Il y a donc \(g_r = 2\), on a donc :
\begin{align*}
	\Omega^{(1-4-4)} = \Omega^{(5-2-2)} &= \dfrac{2!}{1!(2-1)!} \times \dfrac{2!}{2!(1-1)!}\\
	&= \dfrac{2!}{1!} \times \dfrac{2!}{2!0!}\\
	&= 2 \times 1\\
	&= 2
\end{align*}

Il y a donc 4 configurations

\subsubsection{s = 3/2}

Il y a donc \(g_r = 4\), on a donc :
\begin{align*}
	\Omega^{(1-4-4)} = \Omega^{(5-2-2)} &= \dfrac{4!}{1!(4-1)!} \times \dfrac{4!}{2!(4-2)!}\\
	&= \dfrac{4!}{3!} \times \dfrac{4!}{2!2!}\\
	&= 4 \times \dfrac{12}{2}\\
	&= 4 \times 6 = 24
\end{align*}

Il y a donc 24 configurations

\section{Interaction cubique}

\subsection{}

Pour l'énergie : \(12\varepsilon_0\), on prend le sextuplet : (1,1,1,1,2,2), de même pour l'énergie : \(18\varepsilon_0\), on peut utiliser le sextuplet : (2,2,2,2,1,1).
Dans ces deux cas, il faut choisir la position à 2 niveaux  (\(2\varepsilon_0\) pour la boite A et \(\varepsilon_0\) pour la boite B).
On a donc :
\[
	\Omega_A = \Omega_B = C_6^2 = \dfrac{6!}{4! \cdot 2!} = 15
\]

\subsection{}

Les boites sont indépendantes, il y a donc :
\[
	15 \times 15 = 225
\]
états possibles.

\subsection{}

Les énergies possibles pour une particules sont :

\begin{center}

	\begin{tabular}{|c|c|c|c|c|c|c|}
		\hline 
		E(\(\varepsilon_0\) & 3 & 6 & 9 & 11 & 12 & 14 \\ 
		\hline 
		Triplet & 1,1,1 & 1,1,2 & 1,2,2 & 1,1,3 & 2,2,2 & 1,2,3 \\ 
		\hline 
	\end{tabular}

	\begin{tabular}{|c|c|c|c|c|}
		\hline
		E(\(\varepsilon_0\) & 17 & 18 & 19 & 21 \\
		\hline
		Triplet & 2,2,3 & 1,1,4 & 3,3,1 & 1,2,4\\
		\hline
	\end{tabular}

\end{center}

Les seules manières d'obtenir \(30\varepsilon_0\) sur l'ensemble des boites A et B mise en contact sont :
6-24, 9-21, 12-18, 15-15 

Et inversement on peut mettre la boite B les énergies de la boite A.
Les états possibles sont donc :

\begin{center}

	\begin{tabular}{|c|c|c|c|c|c|c|c|}
		\hline 
		\(E_A(\varepsilon_0)\) & 6 & 9 & 12 & 15 & 18 & 21 & 24 \\ 
		\hline 
		\(E_B(\varepsilon_0)\) & 24 & 21 & 18 & 15 & 12 & 9 & 6 \\
		\hline
		\# & 31 & 72 & 225 & 400 & 225 & 72 & 31\\
		\hline 
		P & $\frac{31}{1056}$ & $\frac{72}{1056}$ & 	$\frac{225}{1056}$ & $\frac{400}{1056}$ 	& $		\frac{225}{1056}$ & $\frac{72}	{1056}$ & $	\frac{31}{1056}$ \\ 
		\hline 
	\end{tabular}

\end{center}

\subsection{}

On peut désormais calculer l'énergie moyenne :

\begin{align*}
	&\sum E_A \times P_A = \dfrac{1}{1056} \times \sum E_A \cdot \#\\
	&= \dfrac{1}{1056} \times \left( 31 \cdot 6 + 72 \cdot 9 + 225 \cdot 12 + 400 \cdot 15 + 225 \cdot 18 + 72 \cdot 21 + 31 \cdot 24\right)
\end{align*}

\begin{align*}
	\sum E_A \cdot \# &= 15~840\\
	P &= 15
\end{align*}

\subsection{}

On calcule l'entropie :
\[
	S_f - S_i = k_B \cdot \ln \left( \dfrac{\Omega_f}{\Omega_i} \right)
\]
Avec \(\Omega_f = 1~056\) et \(\Omega_i = 225\), on a donc l'entropie :
\[
	S_f - S_i = k_B \cdot \ln\left(\dfrac{1056}{225}\right)
\]

\subsection{Oscillateur harmonique isotrope}

L'énergie d'un oscillateur harmonique à 3 dimensions est donnée par la relation :
\[
	E = \left( n_x + n_y + n_z + \dfrac{3}{2} \right) \hbar \cdot \omega
\]

Ici on autorise les énergies nulles, ça induit une dégénérescence de 3 dans tout le problème.
Les 5 premiers niveaux d'énergies sont donc :
\[
	E = \frac{3}{2}\hbar \omega,\frac{5}{2}\hbar \omega,\frac{7}{2}\hbar \omega,\frac{9}{2}\hbar \omega,\frac{11}{2}\hbar \omega,
\]



\end{document}